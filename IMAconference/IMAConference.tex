\documentclass[xcolor={table}]{beamer}
\usetheme{Singapore}
\usepackage[utf8]{inputenc}
\usecolortheme{crane}
\usepackage{graphicx}
\usepackage{iwona}
\usepackage{standalone}
\usepackage{tikz}
\usetikzlibrary{arrows}
\usetikzlibrary{decorations.markings}
\usetikzlibrary{calc}
\usetikzlibrary{shapes,snakes}
\usepackage{amsmath}
\usepackage{amsfonts}
\usepackage{amsthm}
\usepackage{mathtools}
\usepackage{minted}
\usemintedstyle{tango}

\definecolor{lightblue}{RGB}{124,190,255}
\definecolor{darkgreen}{RGB}{24,145,0}

\beamertemplatenavigationsymbolsempty
\setbeamerfont{caption}{size=\tiny}


\title
{Using Queueing Network Modelling to Assess the Impact of the OPICP}
\author{Geraint Palmer\newline \scriptsize{Paul Harper, Vincent Knight}}
\date{8th IMA International Conference on Quantitative Modelling in the Management of Health and Social Care}
\titlegraphic{\includegraphics[width=1.5cm]{cflogo}}

\begin{document}
\frame{\titlepage}


% 1st slide, ABUHB, some facs & figures
\begin{frame}
\frametitle{Aneurin Bevan University Health Board}
\begin{figure}
\includegraphics[width=0.45\textwidth]{ABUHBlogo}\\
\includegraphics[width=0.6\textwidth]{Aneurin_Bevan}
\end{figure}
\end{frame}

% 2nd slide, what is OPICP?
\begin{frame}
\frametitle{Older People's Integrated Care Pathway}
\begin{itemize}
  \item Pathway focused around pro-active patient centred coordinated care
  \item Individuals identified through \textit{\textbf{risk stratification}} as being at risk of admission to institutionalised care or becoming frequent users of high cost care
  \item Develop holistic personal \textit{\textbf{Stay Well Plans}} for these individuals, utilising \textit{\textbf{low and no cost services}}
  \item Aim to keep individuals and carers as well and as independent as possible
\end{itemize}
\end{frame}

% 3nd slide, rough map out elderly pathways
% 3rd slide pause, with OPICP, how do rates and flows change, what don't we know and what do we want to investigate?
\begin{frame}
\frametitle{Elderly People's Flows Through Health System}
\end{frame}

% 4th slide, Workforce planning stuff...???
\begin{frame}
\frametitle{Workforce Requirements}
\end{frame}

% 5th slide, Ciw and simulation modelling
\begin{frame}
\frametitle{Simulation with Ciw}
\begin{columns}[c]
\column{0.5\textwidth}
Open Source Python Library

\begin{figure}
\includegraphics[width=0.6\textwidth]{logo}
\end{figure}

{\tiny
\url{https://github.com/geraintpalmer/Ciw}\\
\url{https://pypi.python.org/pypi/Ciw}\\
\url{http://ciw.readthedocs.org}\\}

\column{0.5\textwidth}
\fontsize{8.5pt}{10pt} \inputminted{python}{ciwexample.py}
\end{columns}
\end{frame}


% 6th slide, Aim to keep model general... can be used for other pathways and interventions. May lead to deadlock...
\begin{frame}
\frametitle{Generality}
\end{frame}

% 7th slide, gridlock, intro to deadlock
\begin{frame}
    \frametitle{Deadlock}
    \begin{figure}
    \includestandalone[width=0.7\textwidth]{gridlock}
    \end{figure}
\end{frame}

% 8th slide, digraph and detecting deadlock with knots
\begin{frame}
    \begin{figure}
    \includestandalone[width=\textwidth]{buildupdigraph}
    \end{figure}
\end{frame}

% 9th slide, deadlock detection within Ciw
\begin{frame}
\frametitle{Deadlock Detection in Ciw}
\begin{center}
\fontsize{8.5pt}{10pt} \inputminted{python}{deadlockciw.py}
\end{center}
\end{frame}

% 10th slide, introduce 3 deadlocking networks that will be investigated
\begin{frame}
\frametitle{Three Deadlocking Queueing Networks}
\end{frame}

% 11th slide, Markovian deadlock model 1 Node
\begin{frame}
    \frametitle{Markovian Model of Deadlock}
    \includestandalone[width=\textwidth]{1nodemultiserver}\newline
    \center{\LARGE{$(i)$}}
\end{frame}

\begin{frame}
\center
\scriptsize  \[S = \{i\in\mathbb{N} \nonscript\; | \nonscript\; 0 \leq i \leq n + 2c\}\]
Define $\delta = i_2 - i_1$\newline

\vspace{10 mm}

  $  q_{i_1, i_2} = \left\{
  \begin{matrix*}[ r ]
    \left. \begin{matrix*}[ r ]
      \color{red} \Lambda & \color{red} \text{if } \delta = 1 \\
      \color{blue} (1-r_{11})\mu\text{min}(i, c) & \color{blue} \text{if } \delta = -1 \\
      0 & \text{otherwise}
    \end{matrix*} \right\} & \text{if } i_1 < n + c \\
  \end{matrix*} \right.
$
\vspace{10 mm}

  $q_{i_1, i_2} = \left\{
  \begin{matrix*}[ r ]
    \left. \begin{matrix*}[ r ]
      \color{darkgreen} (c-b)r_{11}\mu & \color{darkgreen} \text{if } \delta = 1 \\
      \color{blue} (1-r_{11})(c-b)\mu & \color{blue} \text{if } \delta = -b-1\\
      0 & \text{otherwise}
    \end{matrix*} \right\} & \text{if } i_1 = n + c + b \\
  \end{matrix*} \right.
  \quad \forall \quad 0 \leq b \leq c$

\vspace{10 mm}

\end{frame}

\begin{frame}
    \begin{figure}
    \includestandalone[width=0.95\textwidth]{MC1nodemultiserv}
    \end{figure}
\end{frame}

\begin{frame}
    \frametitle{Times to Deadlock}
    \includegraphics[width=0.5\textwidth]{varyL_1Nms}
    \includegraphics[width=0.5\textwidth]{varymu_1Nms}\newline
    \includegraphics[width=0.5\textwidth]{varyn_1Nms}
    \includegraphics[width=0.5\textwidth]{varyc_1Nms}
\end{frame}



% 12th slide, Markov deadlock model 2 Node
\begin{frame}
    \frametitle{Markovian Model of Deadlock}
    \includestandalone[width=\textwidth]{2nodemultiserver}\newline
    \center{\LARGE{$(i, j)$}}
\end{frame}


\begin{frame}
\center
\scriptsize \[S = \{(i,j)\in\mathbb{N}^{(n_1+c_1+c_2)\times (n_2+c_2+c_1)} \nonscript\; | \nonscript\; i \leq n_1+c_1+j, \nonscript\; j \leq n_2+c_2+i\}\]

\scalebox{0.8}{\parbox{.5\linewidth}{%
\begin{align*}
  \delta &= (i_2, j_2) - (i_1, j_1)\\
  b_1 &= \max(0, i_1-n_1-c_1)\\
  b_2 &= \max(0, i_2-n_2-c_2)\\
  s_1 &= \min(i_1, c_1)-b_2\\
  s_2 &= \min(i_2, c_2)-b_1\\
\end{align*}
}}


\begin{table}
\begin{center}
\resizebox{\textwidth}{!}{
\begin{tabular}{ l | l | l | l |}
  & $j_1 < n_2 + c_2$ & $j_1 = n_2 + c_2$ & $ j_1 > n_2 + c_2$ \\ \hline
  \rotatebox[origin=c]{90}{$i_1 < n_1 + c_1$} & \begin{tabular}{ l } \color{red} $\Lambda_1$ if $\delta = (1, 0)$ \\ \color{orange} $\Lambda_2$ if $\delta = (0, 1)$ \\ \color{darkgreen} $r_{12}s_1\mu_1$ if $\delta = (-1, 1)$ \\ \color{green} $r_{21}s_2\mu_2$ if $\delta = (1, -1)$ \\ \color{blue} $(1-r_{12})s_1\mu_1$ if $\delta = (-1, 0)$ \\ \color{lightblue} $(1-r_{21})s_2\mu_2$ if $\delta = (0, -1)$ \end{tabular} & \begin{tabular}{ l } \color{red} $\Lambda_1$ if $\delta = (1, 0)$ \\ \color{darkgreen} $r_{12}s_1\mu_1$ if $\delta = (0, 1)$ \\ \color{green} $r_{21}s_2\mu_2$ if $\delta = (1, -1)$ \\ \color{blue} $(1-r_{12})s_1\mu_1$ if $\delta = (-1, 0)$ \\ \color{lightblue} $(1-r_{21})s_2\mu_2$ if $\delta = (0, -1)$ \end{tabular} & \begin{tabular}{ l } \color{red} $\Lambda_1$ if $\delta = (1, 0)$ \\ \color{darkgreen} $r_{12}s_1\mu_1$ if $\delta = (0, 1)$ \\ \color{green} $r_{21}s_2\mu_2$ if $\delta = (0, -1)$ \\ \color{blue} $(1-r_{12})s_1\mu_1$ if $\delta = (-1, 0)$ \\ \color{lightblue} $(1-r_{21})s_2\mu_2$ if $\delta = (-1, -1)$ \end{tabular} \\ \hline
  \rotatebox[origin=c]{90}{$i_1 = n_1 + c_1$} & \begin{tabular}{ l } \color{orange} $\Lambda_2$ if $\delta = (0, 1)$ \\ \color{darkgreen} $r_{12}s_1\mu_1$ if $\delta = (-1, 1)$ \\ \color{green} $r_{21}s_2\mu_2$ if $\delta = (1, 0)$ \\ \color{blue} $(1-r_{12})s_1\mu_1$ if $\delta = (-1, 0)$ \\ \color{lightblue} $(1-r_{21})s_2\mu_2$ if $\delta = (0, -1)$ \end{tabular} & \begin{tabular}{ l } \color{darkgreen} $r_{12}s_1\mu_1$ if $\delta = (0, 1)$ \\ \color{green} $r_{21}s_2\mu_2$ if $\delta = (1, 0)$ \\ \color{blue} $(1-r_{12})s_1\mu_1$ if $\delta = (-1, 0)$ \\ \color{lightblue} $(1-r_{21})s_2\mu_2$ if $\delta = (0, -1)$ \end{tabular} & \begin{tabular}{ l } \color{darkgreen} $r_{12}s_1\mu_1$ if $\delta = (0, 1)$ \\ \color{green} $r_{21}s_2\mu_2$ if $\delta = (1, 0)$ \\ \color{blue} $(1-r_{12})s_1\mu_1$ if $\delta = (-1, 0)$ \\ \color{lightblue} $(1-r_{21})s_2\mu_2$ if $\delta = (-1, -1)$ \end{tabular} \\ \hline
  \rotatebox[origin=c]{90}{$i_1 > n_1 + c_1$} & \begin{tabular}{ l } \color{orange} $\Lambda_2$ if $\delta = (0, 1)$ \\ \color{darkgreen} $r_{12}s_1\mu_1$ if $\delta = (-1, 0)$ \\ \color{green} $r_{21}s_2\mu_2$ if $\delta = (1, 0)$ \\ \color{blue} $(1-r_{12})s_1\mu_1$ if $\delta = (-1, -1)$ \\ \color{lightblue} $(1-r_{21})s_2\mu_2$ if $\delta = (0, -1)$ \end{tabular} & \begin{tabular}{ l } \color{darkgreen} $r_{12}s_1\mu_1$ if $\delta = (0, 1)$ \\ \color{green} $r_{21}s_2\mu_2$ if $\delta = (1, 0)$ \\ \color{blue} $(1-r_{12})s_1\mu_1$ if $\delta = (-1, -1)$ \\ \color{lightblue} $(1-r_{21})s_2\mu_2$ if $\delta = (0, -1)$ \end{tabular} & \begin{tabular}{ l } \color{darkgreen} $r_{12}s_1\mu_1$ if $\delta = (0, 1)$ \\ \color{green} $r_{21}s_2\mu_2$ if $\delta = (1, 0)$ \\ \color{blue} $(1-r_{12})s_1\mu_1$ if $\delta = (-\min(b_1+1,b_2+1), -\min(b_1,b_2+1))$ \\ \color{lightblue} $(1-r_{21})s_2\mu_2$ if $\delta = (-\min(b_1+1,b_2), -\min(b_1+1,b_2+1))$ \end{tabular} \\ \hline
\end{tabular}
}
\end{center}
\end{table}

\end{frame}

\begin{frame}
    \begin{figure}
    \includestandalone[width=0.95\textwidth]{MC2nodemultiserv}
    \end{figure}
\end{frame}

\begin{frame}
    \frametitle{Times to Deadlock}
    \includegraphics[width=0.5\textwidth]{varyL1_2Nms}
    \includegraphics[width=0.5\textwidth]{varymu1_2Nms}\newline
    \includegraphics[width=0.5\textwidth]{varyn1_2Nms}
    \includegraphics[width=0.5\textwidth]{varyc1_2Nms}
\end{frame}


% 13th lide, Markov deadlock model 2 Node self loops
\begin{frame}
    \frametitle{Markovian Model of Deadlock}
    \includestandalone[width=\textwidth]{2nodefeedbackexample}\newline
    \center{\LARGE{$(i, j)$}}
\end{frame}


\begin{frame}
\center
\scriptsize \[S = \{(i,j)\in\mathbb{N}^{(n_1+2\times n_2+2)} \nonscript\; | \nonscript\; 0 \leq i + j \leq n_1 + n_2 + 2\}\cup\{(-1)\}\]
Define $\delta = (i_2, j_2) - (i_1, j_1)$\newline\newline
\tiny{
  $q_{(i_1, j_1),(i_2, j_2)} = \left\{
  \begin{matrix*}[ r ]
    \left. \color{red} \begin{matrix*}[ r ]
      \Lambda_1 & \text{if } i_1 \leq n_1 \\
      0 & \text{otherwise}
    \end{matrix*} \right\} & \color{red} \text{if } \delta = (1, 0)\\
    \left. \color{orange} \begin{matrix*}[ r ]
      \Lambda_2 & \text{if } j_1 \leq n_2 \\
      0 & \text{otherwise}
    \end{matrix*} \right\} & \color{orange} \text{if } \delta = (0, 1) \\
    \left. \color{blue} \begin{matrix*}[ r ]
      (1 - r_{12})\mu_1 & \color{blue} \text{if } j_1 < n_2 + 2 \\
      0 & \text{otherwise}
    \end{matrix*} \right\} & \color{blue} \text{if } \delta = (-1, 0) \\
    \left. \color{lightblue} \begin{matrix*}[ r ]
      (1 - r_{21})\mu_2 & \text{if } i_1 < n_1 + 2 \\
      0 & \color{lightblue} \text{otherwise}
    \end{matrix*} \right\} & \color{lightblue} \text{if } \delta = (0, -1) \\
    \left. \color{darkgreen} \begin{matrix*}[ r ]
      r_{12}\mu_1 & \text{if } j_1 < n_2 + 2 \text{ and } (i_1, j_1) \neq (n_1+2, n_2) \\
      0 & \text{otherwise}
    \end{matrix*} \right\} & \color{darkgreen} \text{if } \delta = (-1, 1) \\
    \left. \color{green} \begin{matrix*}[ r ]
      r_{21}\mu_2 & \text{if } i_1 < n_1 + 2 \text{ and } (i_1, j_1) \neq (n_1, n_2+2)\\
      0 & \text{otherwise}
    \end{matrix*} \right\} & \color{green} \text{if } \delta = (1, -1) \\
    0 & \text{otherwise}
  \end{matrix*} \right.$\newline\newline

  $q_{(i_1, j_1), (-1)} = \left\{
  \begin{matrix*}[ r ]
    \color{magenta!50} r_{11}\mu_1 & \color{magenta!50} \text{if } i > n_1 \text{ and } j < n_2 + 2 \\
    0 & \text{otherwise}
  \end{matrix*}
  \right.$\newline

  $q_{(i_1, j_1), (-2)} = \left\{
  \begin{matrix*}[ r ]
    \color{violet!50} r_{22}\mu_2 & \color{violet!50} \text{if } j > n_2 \text{ and } i < n_1 + 2 \\
    0 & \text{otherwise}
  \end{matrix*}
  \right.$\newline

  $q_{(i_1, j_1), (-3)} = \left\{
  \begin{matrix*}[ r ]
    \color{green} r_{21}\mu_2 & \color{green} \text{if } (i, j) = (n_1, n_2 + 2) \\
    \color{darkgreen} r_{12}\mu_1 & \color{darkgreen} \text{if } (i, j) = (n_1 + 2, n_2) \\
    0 & \text{otherwise}
  \end{matrix*}
  \right.$\newline

$q_{-1, s} = q_{-2, s} = q_{-3, s} = 0$
}
\end{frame}

\begin{frame}
    \begin{figure}
    \includestandalone[width=0.95\textwidth]{markov_chain_feedback}
    \end{figure}
\end{frame}

\begin{frame}
    \frametitle{Times to Deadlock}
    \includegraphics[width=0.5\textwidth]{vary_L1fb}
    \includegraphics[width=0.5\textwidth]{vary_mu1fb}\newline
    \centering
    \includegraphics[width=0.5\textwidth]{vary_n1fb}
\end{frame}

% 14th slide, remind of healthcare motivation... future work
\begin{frame}
\frametitle{Recap about Health Stuff}
\end{frame}


% Thank you slide.
\begin{frame}
    \frametitle{Thank You}
    palmergi1@cardiff.ac.uk
\end{frame}

\end{document}
