\documentclass{beamer}
\usetheme{Singapore}
\usepackage[utf8]{inputenc}
\usecolortheme{crane}
\usepackage{graphicx}
\usepackage{standalone}
\usepackage{tikz}
\usetikzlibrary{arrows}
\usetikzlibrary{decorations.markings}
\usetikzlibrary{calc}
\usetikzlibrary{shapes,snakes}
\usetikzlibrary{decorations.text}
\usepackage{amsmath}
\usepackage{amsfonts}
\usepackage{amsthm}
\usepackage{mathtools}
\usepackage{tcolorbox}
\usepackage{float}
\usepackage{bm}
\usepackage{minted}
\usepackage{booktabs}

\definecolor{lightblue}{RGB}{124,190,255}
\definecolor{darkgreen}{RGB}{24,145,0}
\definecolor{darkorange}{RGB}{220,110,0}
\definecolor{bg}{RGB}{249, 251, 223}

\setbeamertemplate{section in toc}[sections numbered]
\beamertemplatenavigationsymbolsempty
\setbeamerfont{caption}{size=\tiny}


\title{Dulliau ac Offerynnau ar gyfer Ymchwil Ailgynhyrchiadwy}
\author{\textcolor{darkorange}{@GeraintPalmer}}
\date{Cynhadledd Wyddonol 2018}
\titlegraphic{
  \begin{tikzpicture}{width=2.35cm}
    \node at (0, 0) {\includegraphics[height=1.4cm]{ccclogo}};
    \node at (2.5, 0) {\includegraphics[height=1.4cm]{../cflogo}};
    \node at (5.5, 0) {\includegraphics[height=1.1cm]{SSI_Logo}};
  \end{tikzpicture}%
}

\begin{document}

\frame{\titlepage}
\begin{frame}
\frametitle{Ailgynhyrchadwyedd}
\begin{center}
  \textit{``Os gwelais ymhellach na rhai, yna gwnes hynny drwy sefyll ar ysgwyddau cewri o ddynion.''}\\
  \vspace{0.75cm}
  \small{SYR ISAAC NEWTON}
\end{center}
\end{frame}

\begin{frame}
\begin{center}
  \includestandalone[width=0.65\textwidth]{ymchwil}
\end{center}
\end{frame}

\begin{frame}
\frametitle{\hfill}
\tableofcontents
\end{frame}

\section[]{Osgoi llafr llaw}
\begin{frame}
\frametitle{\hfill}
\tableofcontents[currentsection,currentsubsection]
\end{frame}

\begin{frame}
\huge{
  clic clic clic clic clic clic clic clic clic clic clic clic clic clic clic clic clic clic clic clic clic clic clic clic clic clic clic clic clic clic clic clic clic clic clic clic clic clic clic clic clic clic clic clic clic clic clic clic clic clic clic clic clic clic clic clic clic clic clic clic clic clic clic clic clic clic clic clic clic clic clic clic
}
\end{frame}

\begin{frame}
\begin{center}
  \includegraphics[width=0.4\textwidth]{excellogo}
\end{center}
\end{frame}

\begin{frame}[fragile]
\begin{center}
  \includestandalone[width=\textwidth]{sgripto}
\end{center}
\end{frame}

\section[]{Offerynnau agored}
\begin{frame}
\frametitle{\hfill}
\tableofcontents[currentsection,currentsubsection]
\end{frame}

\begin{frame}
  \begin{columns}
    \begin{column}{0.5\textwidth}
      \begin{center}
        \includegraphics[width=0.35\textwidth]{emoji/arian}\\
        Trwyddedau
      \end{center}
    \end{column}
    \pause
    \begin{column}{0.5\textwidth}
      \begin{center}
        \includegraphics[width=0.35\textwidth]{emoji/unlock}\\
        Ffynhonell agored
      \end{center}
    \end{column}
  \end{columns}
\end{frame}

\section[]{Ymarferion cod gorau}
\begin{frame}
\frametitle{\hfill}
\tableofcontents[currentsection,currentsubsection]
\end{frame}

\begin{frame}[fragile]
\scriptsize{
\begin{minted}{python}
a = 1071
b = 462
C = []

while a != 0 and b != 0:
    i = 0
    while a >= b:
        a -= b
        i += 1
    a, b = b, a
    C.append(i)
C.append(a)
C.append(b)

print(max(C))
\end{minted}
}
\end{frame}

\begin{frame}[fragile]
\scriptsize{
\begin{minted}{python}
mwyaf = 1071
lleiaf = 462
ffactorau = []

# Algorithm Euclid
while mwyaf != 0 and lleiaf != 0:
    lluosrif = 0
    while mwyaf >= lleiaf:
        mwyaf -= lleiaf
        lluosrif += 1
    mwyaf, lleiaf = lleiaf, mwyaf  # cyfnewid enwau'r newidynnau
    ffactorau.append(lluosrif)
ffactorau.append(mwyaf)
ffactorau.append(lleiaf)

print(max(ffactorau))
\end{minted}
}
\end{frame}

\begin{frame}[fragile]
\scriptsize{
\begin{minted}{python}
def tynnu_lluosrifau(mwyaf, lleiaf):
    """
    Faint o weithiau mae un rhif yn mynd mewn i rhif arall
    """
    lluosrif = 0
    while mwyaf >= lleiaf:
        mwyaf -= lleiaf
        lluosrif += 1
    return mwyaf, lleiaf, lluosrif

def ffactor_cyffredin_mwyaf(mwyaf, lleiaf):
    """
    Algorithm Euclid y canfod ffactor mwyaf cyffredin dau rhif
    """
    ffactorau = []
    while mwyaf != 0 and lleiaf != 0:
        lleiaf, mwyaf, lluosrif = tynnu_lluosrifau(mwyaf, lleiaf)
        ffactorau.append(lluosrif)
    ffactorau.append(mwyaf)
    ffactorau.append(lleiaf)
    return max(ffactorau)

print(ffactor_cyffredin_mwyaf(1071, 462))
\end{minted}
}
\end{frame}

\begin{frame}[fragile]
\begin{minted}{python}
assert tynnu_lluosrifau(31, 10) == (1, 10, 3)

assert ffactor_cyffredin_mwyaf(1071, 462) == 21
\end{minted}
\end{frame}

\begin{frame}
  \begin{center}
    \vspace{0.8cm}
    \includestandalone[width=0.85\textwidth]{triongl-ymarferion-gorau}
  \end{center}
\end{frame}

\section[]{Rheolaeth fersiwn}
\begin{frame}
\frametitle{\hfill}
\tableofcontents[currentsection,currentsubsection]
\end{frame}

\begin{frame}
\includestandalone[width=\textwidth]{rheolaeth_fersiwn}
\end{frame}

\begin{frame}
\frametitle{Adnoddau Cyfrwng Cymraeg}
\begin{center}
\huge{...ar y ffordd.}\\
\vspace{1cm}
\includegraphics[height=2cm]{ccclogo}\\
\vspace{1cm}
\small{\textcolor{darkorange}{@GeraintPalmer}}
\end{center}
\end{frame}
\end{document}
